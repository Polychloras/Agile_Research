\documentclass[conference]{IEEEtran}
\IEEEoverridecommandlockouts
% The preceding line is only needed to identify funding in the first footnote. If that is unneeded, please comment it out.
\usepackage{cite}
\usepackage{amsmath,amssymb,amsfonts}
\usepackage{algorithmic}
\usepackage{graphicx}
\usepackage{textcomp}
\usepackage{xcolor}
\def\BibTeX{{\rm B\kern-.05em{\sc i\kern-.025em b}\kern-.08em
    \kern-.1667em\lower.7ex\hbox{E}\kern-.125emX}}
\begin{document}

\title{Remote Agile Tools and Technologies: A Comparative Analysis of Remote Collaboration Software\\
}

\author{\IEEEauthorblockN{Humphrey Borketey, Elvis Chukwuani, Kiana Kiashemshaki,}
        \IEEEauthorblockN{Emily Massie, Uchechi Blessing Nwala, and Mehrdad Yadollahi
}

\IEEEauthorblockA{\IEEEauthorrefmark{1}Department of Computer Science\\
Bowling Green State University, Bowling Green, Ohio\\
\{humphbb, elvisc, kkiana, emassie, unwala, mehrday\}@bgsu.edu}
}
\maketitle

\begin{abstract}
    As the agile approach becomes more mainstream to accommodate incremental development, the research into understanding the elements of the agile process becomes more important. Currently, there is little research into how remote agile tools and technologies affect the agile development process. We propose a quantitative study to understand how these technologies affect the agile development process.
\end{abstract}

\begin{IEEEkeywords}
Agile, Remote Communication, Agile Technologies, Software Engineering
\end{IEEEkeywords}

\section{Introduction}
In late 2019, one of the deadliest pandemics struck the world; the COVID-19 pandemic. The rate of contagion grew exponentially, causing it to rapidly spread across the world. The increasing death rate toll led governments and health organizations from all over the world to impose strict measures like lockdowns, travel restrictions, and social distancing. This led to most companies indefinitely closing their physical spaces of operation, resulting in an economic decline. To curtail the damage, businesses adopted a remote where possible. This transition was made possible due to the widespread availability of high-speed internet and advanced conferencing and collaboration tools.

The remote work model generally took the form of 100\% remote work from anywhere or a hybrid approach where professionals would work a specified number of days from anywhere and the rest in person. Software development teams transitioned rapidly to the remote model, adjusting existing agile methodologies to suit their needs [1]. A notable challenge presented by the transition was the reduction in social interactions although this was augmented by activities such as daily standups and retrospective meetings [1].

Agile software development has, over the years, been credited with high productivity [1].  However, the transition to remote work raised questions about productivity. The adoption of advanced conferencing and collaboration tools that mimic the in-person experience has helped in the transition, but to what extent has this impacted productivity as differences exist between remote and in-person workstyles? To understand how these tools impacted the productivity of software development teams, we seek to answer these questions:

How do the technologies used to track milestones and foster communication during software devolpment impact the Agile Process?

What effect do these technologies have on the turnaround time of software built using the agile development process?

How did the adoption rate of remote tools during the pandemic affect existing agile methods?

The rest of this paper is sturctured as follows: Section II (Related works) entails similar papers in this field, their work and the gap our paper fills. Section III (Experiment) goes over how similar research has been conducted. Section IV (Problem and Approach) introduces the problem and talks about our suggested approach. Section V (Target Population) contains information about the test group. Section VI (Survey Questions) goes over the survey questions. Section VII (Technology) basically our survey and how we analyse our responses. Section VIII (Threats to Validity) concerns threats and validity peertaining to this research. Section IX (Conclusion) sections contains summarizes other sections and ends the paper.

\section{Related Work}
As organizations strive to become more adaptable to the dynamic and fast-paced business environments they encounter, the Agile methodology has grown in popularity in recent years [2]. Our research focuses on identifying, evaluating, and comparing various tools and technologies being used by software developers working remotely implementing the Agile software development methodology to develop software. The Agile methodology has occasionally been incorporated into remote software development but due to the pandemic idea of working remotely was forced onto software developers at a large scale without time to be trained or adapt [3]. With a notable shift toward remote work, particularly accelerated by the COVID-19 pandemic, the software development industry has witnessed a fundamental change in how teams collaborate. Agile methodologies, known for their iterative and collaborative approach, have become the gold standard in software development. Understanding how Agile practices adapt to remote work settings has become paramount, as this impacts a significant portion of the workforce [4]. This research focuses on the evaluation and comparison of various tools and technologies employed in Agile software development within the context of remote teams. With the growing prevalence of remote work and the widespread adoption of Agile methodologies, the choice of collaboration tools and technologies has become a critical determinant of the success of distributed software development projects.

Toward the end of the pandemic Ralph et al. [5] conducted research in which questionnaires were sent to multiple remote software developers all over different continents in different sectors, about 2225 valid responses were obtained. His research focused on trying to understand the relationship between the productivity of remote software developers and their well-being, and environment. Results showed some useful correlation, but his research did not focus on specific factors affecting this productivity. Their research just focused on well-being but did not delve into what technologies, agile, and collaboration software were being used which would have led to the decline in productivity. 

In 2021 Mill et al. [6] looked at agile remote software developers by carrying thorough research to gain useful insights into this new stimulus. He conducted a qualitative survey amongst the developers and got about 2265 responses and a quantitative survey with 608 responses. The research was aimed at finding the remote developer’s abilities, to meet set milestones, organizational culture, interaction and cooperation, and communication with stakeholders and other important entities. The results of the study provided evidence that social connection and communication stand out as two of the most noteworthy and prevalent challenges individuals are facing [6]. Their research did not go deep to find out what is causing such problems, it could be resulting from the choice or use of a technology or collaboration software being used in the organization. 

Schmidtner et al. [7] conducted a research study in Germany, the study aimed to gauge the transition to remote software development in software organizations. The research was conducted on supervisors and project management specialists in agile development rather than software developers. The results pointed that despite this unfavorable change in the agile approach setting, the shift to remote work proceeded without an issue. Practitioners are now using online resources they were hesitant to use before the pandemic, and they are experiencing greater flexibility at the expense of a negligible productivity loss [7]. Their research did not explicitly point out which online tools their organization integrated into and what exactly these practitioners did to ensure their smooth transition to benefit other practitioners in our field.

The value of this research lies in its potential to inform decision-making and drive continuous improvement in remote software development. The study's findings will empower software development teams and organizations, offering insights into how the choice of technology, agile, and collaboration software helps agile software developers mitigate remote work challenges, enhancing productivity, communication, and project outcomes. This research not only addresses the current demand for remote work solutions but also provides a pathway for the software development industry to evolve and optimize its practices, delivering better outcomes in the dynamic world of remote software development.

\section{Experiment}
Article [2] provided discusses a study conducted in the context of remote agile work within OP Financial Group, the largest bank in Finland. The study aimed to identify the challenges and solutions related to remote agile work experienced by two teams within the organization during the COVID-19 pandemic and the study focused on two teams within OP Financial Group which had different tasks and work characteristic, so that the researchers assess the impact of remote agile work on teams with varying job requirements. The study likely involved a combination of qualitative research methods, including interviews and observations. Team members and leaders may have been interviewed to gather their perspectives and experiences related to remote agile work. The study likely involved a combination of qualitative research methods, including interviews and observations. Team members and leaders may have been interviewed to gather their perspectives and experiences related to remote agile work. The study concluded by emphasizing the ongoing trends of agile work and remote or hybrid work and the need for organizations to adapt to these changes. The findings aimed to guide organizations in effectively implementing agile practices in a remote work environment.

Article [3] is about “When Agile Harms Learning and Innovation” investigates the impact of Agile methodologies on organizational learning and innovation. The research targets to understand the relationship between Agile practices and learning, explore their influence on innovation, and propose strategies to mitigate negative effects.

The study's methodology includes surveys, interviews, and organizational case studies involving professionals from diverse roles in Agile development. Quantitative data will be analyzed using statistical methods, while qualitative data will be subjected to thematic analysis.

Expected outcomes include the identification of challenges faced by organizations applying Agile, insights into how Agile methodologies affect learning and innovation, and recommendations to address these negative impacts. By offering valuable insights and actionable recommendations, the study contributes to enhancing learning and innovation outcomes in organizations practicing Agile methodologies.

Mill et al. [4] provides insights into scaling agile methodologies within organizations. It discusses various aspects of implementing agile practices across large enterprises. Agile scaling begins with ambitious visions, but leaders should not plan every detail in advance. They should launch an initial wave of agile teams, gather data on their performance, and decide whether to scale further based on the value created. Organizations should create a taxonomy of opportunities, breaking them into components like customer experience teams, business process teams, and technology systems teams. This helps identify potential teams and talent gaps. Prioritizing and sequencing initiatives is crucial. Leaders should consider various criteria, including strategic importance, budget limitations, and risk levels. Transitioning in steps is often more manageable. There is no limit to the number of agile teams an organization can create. Companies can establish "teams of teams" to work on related initiatives. Coordination through daily stand-ups and common rhythms is important. The document emphasizes that agile teams coexist with traditional structures in many organizations. It's crucial to instill agile values and principles across the entire enterprise. This involves changes in values, principles, operating architectures, and organizational models. Scaling agile requires changes in HR procedures to acquire, motivate, and retain talent. HR should align with agile values, evaluate people based on team performance, and provide continuous feedback and coaching. Traditional annual planning and budgeting cycles may not align with agile practices. Funding procedures should be more flexible, resembling those of venture capitalists. Successful scaling of agile leads to more innovation, adaptability, and efficiency. It brings agile values to support functions, improves operating architectures, and delivers measurable improvements in outcomes.

Ralph et al. [5] titled "Pandemic Programming and Its Impact on Software Developers" investigates the effects of the COVID-19 pandemic on software developers and explores strategies for organizational support.

The research aims to understand challenges faced by developers, assess the impact of remote work on productivity and well-being, and identify effective support mechanisms. The study utilizes surveys and interviews with diverse software developers, along with organizational surveys.

Expected outcomes include insights into pandemic-related challenges, productivity impact, and identification of supportive strategies. The research aims to assist organizations in understanding their workforce's needs, fostering productive and supportive environments during crises.

The study [6] is focused on understanding the impact of remote work, particularly working from home (WFH), on software development teams during the COVID-19 pandemic. The researchers conducted surveys to gather data and insights from software developers about their experiences with remote work. The study involved two surveys: the WFH-Survey and the Team-Survey. These surveys were administered to software developers to collect data on various aspects of remote work, team culture, and productivity. The document does not provide specific details about the number or characteristics of survey participants, but it mentions that there were 2,265 responses for the first survey and 608 responses for the second survey. These participants likely include software developers who were working remotely during the COVID-19 pandemic. The primary focus of the research was to understand how remote work, particularly WFH, affected software development teams. This includes examining changes in team culture, social

connections, communication, and productivity. The document discusses the use of data analysis techniques to draw insights from the survey data. The researchers-built models to determine which team factors were most influential in predicting changes in team productivity. The document presents findings related to social connections, communication, and productivity. It notes that many participants reported challenges with social connectedness and communication during remote work. However, most respondents did not perceive a significant change in team productivity. Based on their findings, the researchers provide recommendations for improving social connection and communication in remote software development teams. These recommendations include building and maintaining team culture, including social activities, being mindful of others' time, and actively working to be inclusive. The document mentions the availability of a replication package with surveys and analysis codebook on Zenodo, which can provide more detailed information about the study setup and data collection.

Schmidtner et al. [7] provides diverse Agile teams, including sectors like software development and project management, that transitioned to remote work due to COVID-19. Data collection methods include structured surveys, interviews with Agile team leaders and members, and an analysis of Agile project documentation. Surveys delve into communication, collaboration, project progress, and challenges faced, while interviews provide qualitative insights into specific challenges, solutions, and lessons learned. Thematic analysis is employed to identify common themes in the qualitative data.

The study goal to achieve three primary outcomes. First, it seeks to provide insights into how Agile methodologies were adapted to remote work settings during the pandemic. Second, it seeks to understand the challenges faced by Agile teams in remote work environments and analyze the strategies employed to overcome these challenges. Lastly, the study evaluates the effectiveness of Agile practices in maintaining team collaboration, project progress, and productivity during the pandemic.

By combining quantitative and qualitative data, the research intends to offer a comprehensive understanding of the application of Agile methodologies in remote work contexts. The findings are anticipated to be beneficial for organizations navigating similar challenging circumstances, offering valuable insights to optimize their Agile practices. Additionally, the study provides a foundation for future research in the field of Agile project management.

\section{Problem and Approach}
In the wake of the COVID-19 Pandemic, Businesses underwent major internal changes to adapt to a remote setting, including adopting an agile development approach and using technologies such as GitHub, and Trello. As a result, these technologies have become a part of the agile development process, being used for communication, collaboration, tracking milestones and evaluating progress.  There already exists research that investigates the impact of communication on agile development and the impact of remote working on agile development. However, we found that this research didn’t investigate if these technologies impacted the agile devolvement process. We created a series of research questions in order to create a survey in oder to investigate how the use of agile technology impacts the agile development process.

\section{Target Population}
We were interested in companies that recently switched to an agile process using agile development technologies during their product design. As such we primarily did our research on software engineers that recently switched to an agile approach. We define agile development technologies as a technology or application that can be used to track or manage elements of the agile development process, such as version control software and conferencing applications. 

\section{Survey Questions}
To narrow down the information, we needed to find three main points of context before developing our survey questions. Firstly, a broad understanding of how the company uses these technologies.  Secondly, we needed to know how these technologies were being used to implement the design process. Lastly, there are questions about how these technologies were being used by the employees. Based on these areas, we created the following survey questions: 

\subsection*{Company Survey Questions (SQ) }
SQ 1: What communication technologies have been used by the company between cliets and software developers? 

SQ 2: What milestone tracking technologies have been used by the company during software devlopment? 

SQ 3: How long have these technologies been implemented? 

SQ 4: On average, how many people make up the groups developing software products?

SQ 5: Do employees primarily use these technologies in-person or remotely? 

SQ 6: About how many employees work primarily in Software Development?

\subsection*{Process Survey Questions (SQ)  }
SQ 7: How often are these technologies used to facilitate communication between team members? 

SQ 8: How often are these technologies used to manage milestones during a standard project? 

SQ 9: On a scale of 1 to 10, how reliable is the technology used for communication? 

SQ 10: On a scale of 1 to 10, how reliable is the technology used for milestone management? 


\subsection*{Employee Survey Questions (SQ)  }
SQ 11: How familiar are the employees with Agile development Technologies in general? 

SQ 12: How familiar are the employees with Agile development Technologies implemented by the company? 

SQ 13: How many issues have been raised due to the implementation of this technology? 

SQ 14: How many issues have been resolved due to the implementation of this technology? 

SQ 15: How often are these technologies used for communication purposes? 

SQ 16: How often are these technologies used for milestone management purposes? 

SQ 17: Was any training offered in order to learn how to use these technologies?


 

We then created an anonymous survey broken down into these main parts, the survey contains questions in the form of multiple choice, short answer, and selection.  

\section{Technologies}
We used the Qualtrics survey tool, as it allows for both a simple and freeform construction of our survey and the ability to evaluate the responses. It was important that the responses to our survey remained anonymous, to eliminate any potential personal bias based on the company and preserve the integrity of the employees' responses. 

\section{Threats to Vadility}

Inexperience of a team member: one major threat to validity we identified is that a team member can be inexperienced in using agile methodology for software development. We intend to focus our research on software companies that use agile methodology for the development and implementation of software. Managers or companies can implement these methodologies and tools but not all team members may be experienced in using agile methodology. This can be a major threat to validity as most software companies do not employ developers based on their knowledge of one software development process.
This also can affect the implementation of software if a team member is not very experienced in using agile methodology or tools in software implementation.

Delay in Response time: one major way we intend to communicate is by carrying out a survey, and by sharing questionnaires to software development companies who use agile methodology during implementation of their software. Due to time constraints in carrying out our research, not every company we reached out to would respond to our survey promptly. This will limit our research data and will result in having limited resources for our research.

Limitation in tools: This is a threat to validity as survey questionnaires are not a quality way to measure validity. We intend to also have an in-person interview with the team members of these software companies that use agile methodology during software implementation and design. We also intend to visually represent our data based on information we would collect to derive and give accurate measurements, understanding, and insights into our research.
\section{Conculsion}

As agile development becomes more popular and the need to flexably work grows,it is important to understand how the technologies that are being used to support the Agile process effect the workflow.
We purposed a quantitative research survey to investigate the impact that these agile technologies have on the agile workflow, by looking into its effect on communication and milestone managment. 


\begin{thebibliography}{00}
    \bibitem{b1} M. Neumann et al., "What Remains from Covid-19? Agile Software Development in Hybrid Work Organization: A Single Case Study," 2022 10th International Conference in Software Engineering Research and Innovation (CONISOFT), Ciudad Modelo, San José Chiapa, Mexico, 2022, pp. 29-38, doi: 10.1109/CONISOFT55708.2022.00015.
    \bibitem{b2} R. Reunamäki and C. F. Fey, “Remote agile: Problems, solutions, and pitfalls to avoid,” Bus Horiz, vol. 66, no. 4, pp. 505–516, Jul. 2023, doi: 10.1016/J.BUSHOR.2022.10.003. 
    \bibitem{b3} M. C. Annosi, N. Foss, and A. Martini, “When Agile Harms Learning and Innovation: (and What Can Be Done About It),” Calif Manage Rev, vol. 63, no. 1, pp. 61–80, Nov. 2020, doi: 10.1177/0008125620948265. 
    \bibitem{b4} D. K. Rigby, J. Sutherland, and A. Noble, “Agile at Scale How to go from a few teams to hundreds”.
    \bibitem{b5} P. Ralph et al., “Pandemic programming: How COVID-19 affects software developers and how their organizations can help,” Empir Softw Eng, vol. 25, no. 6, pp. 4927–4961, Nov. 2020, doi: 10.1007/S10664-020-09875-Y/TABLES/7.
    \bibitem{b6} C. Miller, P. Rodeghero, M. A. Storey, D. Ford, and T. Zimmermann, “‘How was your weekend?’ Software development teams working from home during COVID-19,” Proceedings - International Conference on Software Engineering, pp. 624–636, May 2021, doi: 10.1109/ICSE43902.2021.00064. 
    \bibitem{b7} M. Schmidtner, C. Doering, and H. Timinger, “Agile Working During COVID-19 Pandemic,” IEEE Engineering Management Review, vol. 49, no. 2, pp. 18–32, Apr. 2021, doi: 10.1109/EMR.2021.3069940.
\end{thebibliography}

\end{document}
