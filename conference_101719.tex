\documentclass[conference]{IEEEtran}
\IEEEoverridecommandlockouts
% The preceding line is only needed to identify funding in the first footnote. If that is unneeded, please comment it out.
\usepackage{cite}
\usepackage{amsmath,amssymb,amsfonts}
\usepackage{algorithmic}
\usepackage{graphicx}
\usepackage{textcomp}
\usepackage{xcolor}
\def\BibTeX{{\rm B\kern-.05em{\sc i\kern-.025em b}\kern-.08em
    T\kern-.1667em\lower.7ex\hbox{E}\kern-.125emX}}
\begin{document}

\title{Literature Review of Succes and Failure Factors of Agile Methodology in Building Social Media and Mobile Applications\\
{\footnotesize \textsuperscript{*}Note: Sub-titles are not captured in Xplore and
should not be used}
}

\author{
\IEEEauthorblockN{1\textsuperscript{st} Humphrey Borketey}
\IEEEauthorblockA{\textit{Department of Computer Science} \\
\textit{Bowling Green State University}\\
Bowling Green, United States of America \\
humphbb@bgsu.edu}
\and
\IEEEauthorblockN{2\textsuperscript{nd} Elvis Chukwuani}
\IEEEauthorblockA{\textit{Department of Computer Science} \\
\textit{Bowling Green State University}\\
Bowling Green, United States of America \\
elvisc@bgsu.edu}
\and
\IEEEauthorblockN{3\textsuperscript{rd} Kiana Kiashemshaki }
\IEEEauthorblockA{\textit{Department of Computer Science} \\
\textit{Bowling Green State University}\\
Bowling Green, United States of America \\
kkiana@bgsu.ed}
\and
\IEEEauthorblockN{4\textsuperscript{st} Emily Massie}
\IEEEauthorblockA{\textit{Department of Computer Science} \\
\textit{Bowling Green State University}\\
Bowling Green, United States of America \\
emassie@bgsu.edu}
\and
\IEEEauthorblockN{5\textsuperscript{nd} Uchechi Blessing Nwala}
\IEEEauthorblockA{\textit{Department of Computer Science} \\
\textit{Bowling Green State University}\\
Bowling Green, United States of America \\
unwala@bgsu.edu}
\and
\IEEEauthorblockN{6\textsuperscript{rd} Mehrdad Yadollahi}
\IEEEauthorblockA{\textit{Department of Computer Science} \\
\textit{Bowling Green State University}\\
Bowling Green, United States of America \\
mehrday@bgsu.edu}
}

\maketitle

\begin{IEEEkeywords}
component, formatting, style, styling, insert
\end{IEEEkeywords}

\section{Introduction}
This document is a model and instructions for \LaTeX.
Please observe the conference page limits. 

\section{Literture Review}

\subsection{Common Factors used in Agile Development of Mobile Apps}

Through our research we have discovered that current software devolpment of both social media and mobile applications
use an Agile approach due to its advantages during a long-term project. Agile development provides an iteterative approach
that doesn't require an exstensive knowledge of all features currently used in the inital design. Agile is commonally described as a flexable software
devolpment process. The success of an software application is defined by [2] how well the application meets the users needs and requirments.
Therefore, most studies measure the Success or Failure of the factors that impact agile development
through quantitative surveys that aim to determine the satisfaction of the users and intention of the application[1][5][9][10][11].
So far the factors that have been researched through quantitative reasearch based on their effect on production quality include Time-To-Market, Uncertentity, 
New-Product-Development-Process (NPD), Cost of Delay, Team Dynamics, Technologies, and Enviromental Factors. [3][9] 

Team Dynamics

\section{The Factor we are looking at}


\subsection{Common trends in research: technologies, assumptions, qantitative questions}\label{AA}


\subsection{Motivation}


\subsection{Gaps}

\begin{thebibliography}{00}
    \bibitem{b1} A. Rahmat and N. A. M. Hanifiah, ``Usability Testing in Kanban Agile Process for Club Management System," 2020 6th International Conference on Interactive Digital Media (ICIDM), Bandung, Indonesia, 2020, pp. 1-6, doi: 10.1109/ICIDM51048.2020.9339668. 
    \bibitem{b2} P. Jain, A. Sharma and L. Ahuja, ``The Impact of Agile Software Development Process on the Quality of Software Product," 2018 7th International Conference on Reliability, Infocom Technologies and Optimization (Trends and Future Directions) (ICRITO), Noida, India, 2018, pp. 812-815, doi: 10.1109/ICRITO.2018.8748529. 
    \bibitem{b3} P. Jain, A. Sharma and L. Ahuja,``The Model for Determining Weight Coefficients of Maintainability Criteria in Agile Software Development Process," 2019 4th International Conference on Internet of Things: Smart Innovation and Usages (IoT-SIU), Ghaziabad, India, 2019, pp. 1-4, doi: 10.1109/IoT-SIU.2019.8777609. 
    \bibitem{b4} M. A. Islam, R. Hasan and N. U. Eisty, ``Documentation Practices in Agile Software Development: A Systematic Literature Review," 2023 IEEE/ACIS 21st International Conference on Software Engineering Research, Management and Applications (SERA), Orlando, FL, USA, 2023, pp. 266-273, doi: 10.1109/SERA57763.2023.10197828. 
    \bibitem{b5} I. Fatema and K. Sakib, ``Factors Influencing Productivity of Agile Software Development Teamwork: A Qualitative System Dynamics Approach," 2017 24th Asia-Pacific Software Engineering Conference (APSEC), Nanjing, China, 2017, pp. 737-742, doi: 10.1109/APSEC.2017.95. 
    \bibitem{b6} V. Patil, S. Panicker, and M. K. V, ``Use of Agile Methodology for Mobile Applications," International Journal of Latest Technology in Engineering, Management \& Applied Science (IJLTEMAS), vol. 5, no. 10, pp. 73, Oct. 2016, ISSN 2278-2540.
    \bibitem{b7} P. Khumwichai, P. Ratnapinda and W. Sarachai, ``Implementing Information Technology and Social Media for Promoting Tourism Pongyeang Subdistrict, Chiang Mai, Thailand," 2019 16th International Conference on Electrical Engineering/Electronics, Computer, Telecommunications and Information Technology (ECTI-CON), Pattaya, Thailand, 2019, pp. 57-60, doi: 10.1109/ECTI-CON47248.2019.8955137.
    \bibitem{b8} M. Hamdani and W. H. Butt, ``Success and Failure Factors in Agile Development," 2017 International Conference on Computational Science and Computational Intelligence (CSCI), Las Vegas, NV, USA, 2017, pp. 981-986, doi: 10.1109/CSCI.2017.169.
    \bibitem{b9} E. Van Kelle, J. Visser, A. Plaat and P. van der Wijst, ``An Empirical Study into Social Success Factors for Agile Software Development," 2015 IEEE/ACM 8th International Workshop on Cooperative and Human Aspects of Software Engineering, Florence, Italy, 2015, pp. 77-80, doi: 10.1109/CHASE.2015.24.
    \bibitem{b10} S. A. Ajila, ``Have the Factors Affecting Software New Product Development (S-NPD) Changed in the Age of Mobile Apps and Agile Methods?," 2023 Portland International Conference on Management of Engineering and Technology (PICMET), Monterrey, Mexico, 2023, pp. 1-8, doi: 10.23919/PICMET59654.2023.10216803.
    \bibitem{b11} Fahad S. Altuwaijri, Maria Angela Ferrario, ``Factors affecting Agile adoption: An industry research study of the mobile app sector in Saudi Arabia,
    Journal of Systems and Software,"Volume 190, 2022, 111347, ISSN 0164-1212, https://doi.org/10.1016/j.jss.2022.111347. (https://www.sciencedirect.com/science/article/pii/S016412122200084X)
    \bibitem{b12} İ. Cereci and Z. Karakaya, ``Need for a Software Development Methodology for Research-Based Software Projects," 2018 3rd International Conference on Computer Science and Engineering (UBMK), Sarajevo, Bosnia and Herzegovina, 2018, pp. 648-651, doi: 10.1109/UBMK.2018.8566613.
    \bibitem{b13} A. R. Chaudhari and S. D. Joshi, ``Study of effect of Agile software development Methodology on Software Development Process," 2021 Second International Conference on Electronics and Sustainable Communication Systems (ICESC), Coimbatore, India, 2021, pp. 1-4, doi: 10.1109/ICESC51422.2021.9532842.
    \end{thebibliography}

\end{document}
