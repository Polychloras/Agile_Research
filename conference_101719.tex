\documentclass[conference]{IEEEtran}
\IEEEoverridecommandlockouts
% The preceding line is only needed to identify funding in the first footnote. If that is unneeded, please comment it out.
\usepackage{cite}
\usepackage{amsmath,amssymb,amsfonts}
\usepackage{algorithmic}
\usepackage{graphicx}
\usepackage{textcomp}
\usepackage{xcolor}
\def\BibTeX{{\rm B\kern-.05em{\sc i\kern-.025em b}\kern-.08em
    \kern-.1667em\lower.7ex\hbox{E}\kern-.125emX}}
\begin{document}

\title{Remote Agile Tools and Technologies: A Comparative Analysis of Remote Collaboration Software\\
}

\author{\IEEEauthorblockN{Humphrey Borketey\IEEEauthorrefmark{1}, Elvis Chukwuani\IEEEauthorrefmark{2}, Kiana Kiashemshaki\IEEEauthorrefmark{3}},
        \IEEEauthorblockN{ Emily Massie\IEEEauthorrefmark{4}, Uchechi Blessing Nwala\IEEEauthorrefmark{5} and Mehrdad Yadollahi\IEEEauthorrefmark{6}
}

\IEEEauthorblockA{\IEEEauthorrefmark{1}Department of Computer Science\\
Bowling Green State University, Bowling Green, Ohio\\
\{humphbb, elvisc ,kkiana ,emassie, unwala, mehrday\}@bgsu.edu}
}
\maketitle

\begin{IEEEkeywords}
Agile, Remote Communication, Agile Technologies, Software Engineering
\end{IEEEkeywords}

\section{Literature Review}
As organizations strive to become more adaptable to the dynamic and fast-paced business environments they encounter, the Agile methodology has grown in popularity in recent years [1]. Our research focuses on identifying, evaluating, and comparing various tools and technologies being used by software developers working remotely implementing the Agile software development methodology to develop software. The Agile methodology has occasionally been incorporated into remote software development but due to the pandemic idea of working remotely was forced onto software developers at a large scale without time to be trained or adapt [2]. With a notable shift toward remote work, particularly accelerated by the COVID-19 pandemic, the software development industry has witnessed a fundamental change in how teams collaborate. Agile methodologies, known for their iterative and collaborative approach, have become the gold standard in software development. Understanding how Agile practices adapt to remote work settings has become paramount, as this impacts a significant portion of the workforce [3]. This research focuses on the evaluation and comparison of various tools and technologies employed in Agile software development within the context of remote teams. With the growing prevalence of remote work and the widespread adoption of Agile methodologies, the choice of collaboration tools and technologies has become a critical determinant of the success of distributed software development projects.

Towards the end of the pandemic Ralph et al.[4] conducted research in which questionnaires were sent to multiple remote software developers all over different continents in different sectors, about 2225 valid responses were obtained. His research focused on trying to understand the relationship between the productivity of remote software developers and their well-being, and environment. Results showed some useful correlation, but his research did not focus on specific factors affecting this productivity. Their research just focused on well-being but did not delve into what technologies, agile, and collaboration software were being used which would have led to the decline in productivity. In 2021 Mill et al.[5] looked at agile remote software developers by also carrying well thorough research to gain some useful insights into this new stimulus. He conducted a qualitative survey amongst the developers and got about 2265 responses and a quantitative survey with 608 responses. The research was aimed at finding the remote developer’s abilities, to meet set milestones, organizational culture, interaction and cooperation, and communication with stakeholders and other important entities. The results of the study provided compelling evidence that social connection and communication stand out as two of the most noteworthy and prevalent challenges individuals are facing[5]. Their research did not go deep to find out what is causing such problems, it could be resulting from the choice or use of a technology or collaboration software being used in the organization. 

Schmidtner et al.[6] conducted a research study in Germany, the study aimed to gauge the transition to remote software development in software organizations. The research was conducted on supervisors and project management specialists in agile development rather than software developers. The results pointed that there was an adverse change in the agile approach also that the shift to remote work proceeded without a hitch, practitioners are now using online resources that they were hesitant to use before the epidemic, and they are experiencing greater flexibility at the expense of a negligible productivity loss[6]. Their research did not explicitly point out which online tools their organization integrated into and what exactly these practitioners did to ensure their smooth transition to benefit other practitioners in our field.

The value of this research lies in its potential to inform decision-making and drive continuous improvement in remote software development. The study's findings will empower software development teams and organizations, offering insights into how the choice of technology, agile, and collaboration software helps agile software developers mitigate remote work challenges, enhancing productivity, communication, and project outcomes. This research not only addresses the current demand for remote work solutions but also provides a pathway for the software development industry to evolve and optimize its practices, delivering better outcomes in the dynamic world of remote software development.


\begin{thebibliography}{00}
    \bibitem{b1} R. Reunamäki and C. F. Fey, “Remote agile: Problems, solutions, and pitfalls to avoid,” Bus Horiz, vol. 66, no. 4, pp. 505–516, Jul. 2023, doi: 10.1016/J.BUSHOR.2022.10.003. 
    \bibitem{b2} M. C. Annosi, N. Foss, and A. Martini, “When Agile Harms Learning and Innovation: (and What Can Be Done About It),” Calif Manage Rev, vol. 63, no. 1, pp. 61–80, Nov. 2020, doi: 10.1177/0008125620948265. 
    \bibitem{b3} D. K. Rigby, J. Sutherland, and A. Noble, “Agile at Scale How to go from a few teams to hundreds”.
    \bibitem{b4} P. Ralph et al., “Pandemic programming: How COVID-19 affects software developers and how their organizations can help,” Empir Softw Eng, vol. 25, no. 6, pp. 4927–4961, Nov. 2020, doi: 10.1007/S10664-020-09875-Y/TABLES/7.
    \bibitem{b5} C. Miller, P. Rodeghero, M. A. Storey, D. Ford, and T. Zimmermann, “‘How was your weekend?’ Software development teams working from home during COVID-19,” Proceedings - International Conference on Software Engineering, pp. 624–636, May 2021, doi: 10.1109/ICSE43902.2021.00064. 
    \bibitem{b6} M. Schmidtner, C. Doering, and H. Timinger, “Agile Working During COVID-19 Pandemic,” IEEE Engineering Management Review, vol. 49, no. 2, pp. 18–32, Apr. 2021, doi: 10.1109/EMR.2021.3069940.
\end{thebibliography}

\end{document}
